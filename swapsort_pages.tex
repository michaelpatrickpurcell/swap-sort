\documentclass[a6paper, parskip=half, DIV=14, 10pt]{scrartcl}
\usepackage{swapsort}
\usepackage{booktabs}
\usepackage{multicol}
\setlength\columnsep{3em}
\usepackage{enumitem}
\usepackage{caption}
\usepackage{scrlayer-scrpage} % Manage headers and footers in Koma-Script document classes
\setlength{\footskip}{1cm}

\usepackage[type={CC}, version={4.0}, modifier={by}]{doclicense} % Add text and icons for creative commons license
\usepackage{array}
\usepackage{afterpage}

\usepackage[hidelinks]{hyperref} % Add hyperlinks to the pdf file. This should usually be the last package loaded before \begin{document}

\setmainfont{Roboto}
\makeatletter
\newcommand{\version}[1]{\newcommand{\@version}{#1}}
\makeatother

% Set header
\clearpairofpagestyles
\makeatletter
\cfoot*{\normalshape Version \@version}
\makeatother

% Minimize unwanted hyphenation
\tolerance=1
\emergencystretch=\maxdimen
\hyphenpenalty=10000
\hbadness=10000

\setkomafont{section}{\setmainfont{Roboto Slab}\Large\bfseries}
\setkomafont{subsection}{\setmainfont{Roboto Slab}\large\bfseries}
\setkomafont{subsubsection}{\setmainfont{Roboto Slab}\normalsize\bfseries}
\setkomafont{descriptionlabel}{\setmainfont{Roboto}\normalsize\bfseries}

\RedeclareSectionCommand[
  runin=false,
  afterindent=false,
  beforeskip=1ex,
  afterskip=0ex,
]{section}

\RedeclareSectionCommand[
  runin=false,
  afterindent=false,
  beforeskip=1ex,
  afterskip=0ex,
]{subsection}

\RedeclareSectionCommand[
  runin=false,
  afterindent=false,
  beforeskip=1ex,
  afterskip=0ex,
]{subsubsection}

\newcommand{\textRN}[1]{{\setmainfont{Roboto Slab} \RomanNumeral{#1}}}

\newcommand{\card}[1]{{\setmainfont{Roboto Slab} #1}}

\version{1.0}
\begin{document}
\setmainfont{Roboto}%
\raggedright%

\begin{center}
{
\setmainfont[Scale=1.58]{Roboto Slab-Bold}
\Huge
SWAP MEET
}
\end{center}

\section*{Overview}
This is a cooperative game for three to six players.\\It can be played in about ten minutes and is intended for players who are at least eight years old.

\section*{Components}
\begin{description}[leftmargin=0pt, labelsep=\widthof{\ }]
	\item[Numbered Cards (18) \textendash] One card for each whole number between one and eighteen.
\end{description}

\section*{Setup}
\begin{enumerate}[leftmargin=*]
	\item Shuffle all of the cards together. 
	\item Deal one card face down to each player. Everyone should look at their card and add it to their hand. 
	\item Place six cards face down in a single row in the middle of the table. 
	\item Discard the remaining cards. Do not look at their faces. You won't use these cards during the game.
	\item Randomly choose a player to go first. Thereafter, play proceeds to the left (i.e. clockwise).
\end{enumerate}

\newpage

\section*{Playing the Game}
Sort the cards on the table into either ascending or descending order.
To do so, take turns swapping the card in your hand with a card on the table.

Cards are always placed on the table face down.\\
You may not look at the face of any card on the table.
You may look at the face of the card in your hand.

Communication between players is strictly limited.\\In particular, you may never speak during the game.
You may only communicate by indicating whether you believe the cards on the table have been sorted.
This should be done using a simple nonverbal signal such as a ``thumbs up'' gesture.

The game ends when everyone indicates that they believe the cards have been sorted.

To see if you are correct, turn all of the cards on the table face up.
If the cards on the table have indeed been sorted, you win!
Otherwise, you lose.




\vfill
\hrulefill

\textbf{Game Design}: Michael Purcell\\
\textbf{Contact}: \href{mailto:mike@armiger.games}{mike@armiger.games}\\
\begin{tabular}{@{}m{\columnwidth-\widthof{\Huge{\doclicenseIcon}}-0.5cm}@{\hspace{0.05cm}}m{\widthof{\Huge{\doclicenseIcon}}}@{}}
{\textbf{License}: This work is licensed\newline under a ``CC BY 4.0'' license.} & \Huge{\doclicenseIcon}\\
\end{tabular}

\end{document}
